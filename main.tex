\documentclass[twoside,twocolumn]{article}
\usepackage[utf8]{inputenc}
\usepackage[T2A]{fontenc}						% кодировка
\usepackage[english,russian]{babel}				% локализация и переносы
\usepackage{color}
%\usepackage[sc]{mathpazo} 					% Use the Palatino font
%\linespread{1.05} 							% Line spacing - Palatino needs more space between lines
%\usepackage{microtype} 					% Slightly tweak font spacing for aesthetics
\usepackage[hmarginratio=1:1,top=32mm,columnsep=20pt]{geometry} 	% Document margins

\usepackage{fancyhdr} 					% Headers and footers
\pagestyle{fancy} 						% All pages have headers and footers
\fancyhead{} 							% Blank out the default header
\fancyfoot{} 							% Blank out the default footer
\fancyhead[C]{Тайна кургана алебастровых костей$\bullet$Dungeons\&Dragons 5ed}
\fancyfoot[RO,LE]{\thepage} 			% Custom footer text

\usepackage{titlesec} 					% Allows customization of titles
\renewcommand\thesection{\Roman{section}} % Roman numerals for the sections
\renewcommand\thesubsection{\roman{subsection}} % roman numerals for subsections
\titleformat{\section}[block]{\large\scshape\centering}{\thesection.}{1em}{} % Change the look of the section titles
\titleformat{\subsection}[block]{\large}{\thesubsection.}{1em}{} % Change the look of the section titles

\usepackage[hang, small,labelfont=bf,up,textfont=it,up]{caption} % Custom captions under/above floats in tables or figures
\usepackage{booktabs} 					% Horizontal rules in tables

\usepackage{enumitem} 					% Customized lists
\setlist[itemize]{noitemsep} 			% Make itemize lists more compact

\usepackage{titling} 					% Customizing the title section

\usepackage{abstract} 											% Allows abstract customization
%\renewcommand{\abstractnamefont}{\normalfont\bfseries} 			% Set the "Abstract" text to bold
\renewcommand{\abstracttextfont}{\normalfont\small\itshape} 	% Set the abstract itself to small italic text

%----------------------------------------------------------------------------------------
%	TITLE SECTION
%----------------------------------------------------------------------------------------

\setlength{\droptitle}{-4\baselineskip} % Move the title up
\pretitle{\begin{center}\Huge\bfseries} % Article title formatting
\posttitle{\end{center}} 				% Article title closing formatting
\title{Тайна кургана алебастровых костей} % Article title
% \author{Мастер Толедо}					% Your name
\author{\textsc{Конвент Мастеров "Львы и Драконы"}\thanks{Для группы начинающих игроков} \\[1ex] \\ % Your institution
\normalsize {Мастер Толедо}
}
\date{} 								% Leave empty to omit a date. Я убрал \today
\renewcommand{\maketitlehookd}{%
\begin{abstract}
\noindent События происходят в далеких лесах Севера, в землях, что окружают Ривский Залив севернее Камней Саламандр. Кальдера древнего вулкана приютила загадочную Долину Творения. В наши дни редкому интересующемуся это место известно как Выгребные Ямы Глива. Небольшой город Вирискалле находится на самой границе Зеленых Равнин, за ним можно встретить только деревни, которые разбежались по равнинам на два-три дня пути друг от друга. С недавнего времени волшебник Иштван со всеми удобствами разместился в центре Долины и нашел свой способ обеспечить себя работниками и охранниками, чтобы, следуя только одному ему известным целям, проводить свои эксперименты над самым неожиданным материалом - грязью сих Выгребных Ям.
\end{abstract}
}

%-----------------
\begin{document}
\renewcommand{\abstractname}{\vspace{-\baselineskip}}

\maketitle

\section{Подготовка к игре}
Сценарий для группы из трех - пяти приключенцев. Игрокам предлагается создать персонажей третьего утровня до начала приключения. Для участия в приключении знакомство с миром не требуется.

\section{Персонажи Мастера}
\textbf{Иштван Сеченьи} - выглядит как человек старше 50 лет, крепкого телосложения, высокого роста, одет в балахон очень дорогого пошива, ведет себя как тот, кто уверен в своем праве на самое разнузданное эпикурейство. В разговоре растягивает гласные, словно смакуя их на вкус. Постоянно подчеркивает, что все должно по правилам и согласно договору.

\textbf{Дилвиц Оларендельт} - имп, пол неизвестен, говорит и относится ко всему с нескрываемым сарказмом, как чужак, которому безразлично происходящее здесь и сейчас. На шее носит очень элегантный ошейник. Магически подчинен и расположен к Иштвану.  

\textbf{Майилла Мегфигелише} - старейшина деревни Куллика, пожилая женщина в бело-сером платье, уверена в том, что Грязь-из-Ям может исцелять любые болезни и ищет способ раздобыть достаточно Грязи-из-Ям. 

\textbf{Энеро Шойом} - защитник деревни Куллика, командует Соколами деревни Куллики, терпеть не может Иштвана и совершенно не заинтересован в Грязи-из-Ям. 

\section{Сценарий}

\subsection{Встреча героев}

Мастер - героям: 
"Вы в городе Вирискалле. В том самом Вирискалле, про который шутят, мол, не город, а скотопригон, мол, поехал туда за подарками, да привез три конских хвоста, два бараньих курдюка, да одно коровье вымя. И как не пошутить, если всем известно, что весь скот с Больших Равнин ведут продавать сюда. Каждый в городе так или иначе связан с этой торговлей, либо пригонщик, либо торговец, либо охранник. Если лекарь - то лечи двуногих и четвероногих, если жрец - то умей благословлять не считая ног, если конторщик - то носи сапоги с пришивным голенищем, улицы известно чем укрыты. Хотя в последнее время появился в Вирискалле новый народ, из тех, кому на месте не сидится - все бы колобродить, но долго в городе они не засиживаются и часто без следов исчезают на Больших Равнинах. А запах? Настоящий вирискаллец его не замечает, а прибывшие не в счет".

Мастер - героям: 
"Чем вы занимались в городе после своего прибытия, если работали (есть деньги и авторитет), если помогали (есть немного денег, есть авторитет и поручители), если бездельничали (денег нет, поручителей нет, авторитета нет)?". "Как получилось, что вы оказались здесь с совершенно пустым кошельком?". Если герои потратили время на общественно полезные занятия, то им можно получить бесплатно точную карту (сократит путь и покажет ясно окрестности), зелье (см. недорогие целебные зелья), благословение (advantage или см. Core Rules DSA).

В городе есть канцелярия, рынок, храм, лавка снадобий, таверна. Город охраняет милиция под командованием молодого десятского Альмуша Шигудру. 

Мастер - героям: "".
Таверна "Пегий путорак" и ее хозяйка Магда Тепеш, вдова Чабы Пегого.

\subsection{Интересные слухи}
Набор проверок для героев.
\begin{itemize}
\item \textcolor{blue}{Проверка 7} Нужны руки для уборке навоза за стадами, которых перегоняют с равнин в Риву через Вирискалле, по словам местных, это все, чем можно заняться.
\item \textcolor{blue}{Проверка 9} Официантка Кшеся жалуется, что ее молодой человек по имени Бирка ушел из Вирискалле с несколькими молодыми людьми с торговым караваном, что проходил пять дней назад. По словам Кшеси, "этот краснорожий боров - торговец каравана" во всю хвастал товаром, который ожидает с большой выгодой продать все товары в Долине волшебнику Иштвану. Показывал товары Кшеси, и ей очень понравилось светло-зеленое сукно с цветочным узором, все действительно было слишком дорого для местных кошельков.
\item \textcolor{blue}{Проверка 11} Местный охотник по прозвищу Выдра рассказывает, что из деревни, что дальше по тракту, народ давно ходит на заработки в Долину к Иштвану, потом их видят шумно гуляющими в Риве. Некоторые из местных подтверждают, что даже помнят эти пьянки и похмелье после них. 
\end{itemize}

если нам нужны дополнительные пояснения - использовать это блок\footnote{пример пояснения}.

если нам нужна цитата - использовать эту маркировку \cite{Core:2016}.

\textcolor{red}{Для глаз Мастера}

\subsection{Спускаясь в Долину}
Мастер - героям:
"Если бы вам кто-то принялся рассказывать в "Пегом путораке" о том, как выглядит Долина, вы бы подняли на смех говоруна, как несущего откровенную чушь. Но в этот миг, когда калдера Долины открылась перед вами".

\subsection{На подходе к палаццо Сеченьи}
Мастер - героям:
"---".

\subsection{Сбор грязи. Дилвиц}
Мастер - героям:
"---".

\subsection{Волшебный ужин}
Мастер - героям:
"---".

\section{Материалы}
\begin{table}
\caption{Камни из Долины}
\centering
\begin{tabular}{llr}
\toprule
\multicolumn{2}{c}{Название} \\
\cmidrule(r){1-2}
Цвет & Вид & Цена \\
\midrule
Розовый & аметист & $15$ \\
Рыжий & аквамарин & $20$ \\
Светлый & хризолит & $35$ \\
Молочный & нефрит & $45$ \\
Переливчатый & аметрин & $25$ \\
Медовый & янтарь & $50$ \\
Рябиновый & сердолик & $35$ \\
Голубой & кварц & $20$ \\
\bottomrule
\end{tabular}
\end{table}

\subsection{Карта Авентурии}
Найти карту с нормальным разрешением

\subsection{Мужские имена}
Калош, Эстергом, Видин, Яначи, Берток, Габор, Винс, Джани, Джеза, Имр, Калман, Лексо, Морик, Нандор, Пети, Роби, Тибор, Фабо, Ференк

\subsection{Женские имена}
Аранка, Вилма, Дора, Илка, Клара, Панни, Рената, Сари, Франси

%----------------------------------------------------------------------------------------
%	REFERENCE LIST
%----------------------------------------------------------------------------------------

\begin{thebibliography}{99} % Bibliography - this is intentionally simple in this template

\bibitem[Core Rulebook]{Core:2016}
\newblock The Dark Eye english version.
\newblock {\em Core Rulebook}, 414.
 
\end{thebibliography}

%----------------------------------------------------------------------------------------

\end{document}
